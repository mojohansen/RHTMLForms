\documentclass[notitlepage,11pt]{article}
%twocolumn
\usepackage{epsfig,psfig}
\usepackage{fullpage,psfrag}
\usepackage{graphicx}
\usepackage{color}
\usepackage{subfigure}
\usepackage{rotating}
\usepackage{setspace}
\usepackage{geometry}
\usepackage{endfloat}
\usepackage{hyperref}


%\usepackage{loflot}
%\setstretch{0.8}

%\textheight 8.5in
%\oddsidemargin 0.03in
%\textwidth 6.0in
%\topmargin -.25in
%\parskip .1in


%FIRST DRAFT: July 01, 2004

%%%%%%%%%%%%%%%%%%%%%%%%%%%%%%%
%Manipulating the \url comment
\usepackage{url}

%% Define a new 'leo' style for the package that will use a smaller font.
\makeatletter
\def\url@leostyle{%
  \@ifundefined{selectfont}{\def\UrlFont{\sf}}{%
    \def\UrlFont{\small\ttfamily}}\Url@do
}
\makeatother
%%%%%%%%%%%%%%%%%%%%%%%%%%%%%%%%

\title{Automating Access to Data from HTML Forms with Applications in Genomics}
\author{}


\date{\today}
\usepackage{natbib}


\begin{document}

\urlstyle{leo}
\maketitle


\begin{abstract}
\end{abstract}

\section{Introduction}
Functional genomics field is gaining acceleration as more 
data sources from multiple organisms are becoming available. Statistical analysis of genomic data typically  requires 
access to various online genomic databases either for main analysis or during  downstream analysis for 
validation purposes or for both. A large collection of R annotation tools exists for using   during analysis 
of  DNA microarray  gene expression experiments as a part of Bioconductor Project (give ref, e.g., AnnBuilder, annotate, and annaffy
 packages). Among these  tools are R functions that provide querying support for different web services provided by 
National Library of Medicine (NLM) and  the National Center for Biotechnology Information (NCBI). 


Currently, analysis of important sources of genomic data such as ChIP-Chip data \citep{Cawley04, Keles04.2}, 
comparative genomic data include
downloading data via HTML forms. This can be tedious and error prone when done repeatedly. Since the data sources are 
constantly evolving, i.e., genome assemblies are being updated, there is a need to repeat the downloading  
process as new versions of the data become available.  In the case of human ChIP-Chip data \citep{Cawley04, Keles04.2}, for example,  after having 
identified bound regions on the genome, 
i.e., potential targets of the transcription factor of interest, by analyzing the experimental data,  the next step involves 
extracting  
DNA sequences that are in the immediate 
vicinity of  these regions to search for regulatory motifs, i.e., DNA binding sites. Moreover, information as to where these
 regions are  located, i.e., near CpG islands, within or around known (annotated) genes, also gain importance hence one needs  access
to the locations of these sequence features so that orientation of the identified bound regions with respect to  these
 features can be identified.
Another important example that typically requires downloading data via HTML forms arises in the field of comparative genomics. 
It is often interest 
to take a co-expressed group of genes and then analyze their sequence  data together with corresponding orthologus sequences 
from related species. The orthologus information is typically  downloadable via a HTML form from a genomic web site.
In this article, we provide programmatic access to data that are available for downloading  via HTML forms. We achieve this
through a new R package that we developed.  Using this package, users can create R functions corresponding to their specified 
HTML forms. These  HTML form specific R functions allow users to access the corresponding online databases by inputting similar 
elements to the form, that they would have inputted to the HTML form, without ever leaving the R environment.
Aside from automated access to  data, this allows programmatic manipulation of  the obtained data.


This paper is organized as follows. On the next section, we discuss the motivation of our approach in detail and 
outline general principles and components of it. Sections \ref{SA} and \ref{ID} focus on characteristics and 
implementation of our approach. In Section \ref{EG}, we provide several examples, we firstly described how the data described 
in each example  are accessed via HTML forms and then illustrate our automated access to these data. This is followed by 
a summary and discussion of our approach.



\section{Motivation} \label{M} 

***Motivation for programmatic access and why this very sensible and natural to do-extraction 
from Duncan's write up ``Automating Access to Data from HTML Forms''.***

\section{Software architecture} \label{SA}
***Describe components, characteristics, and philosophy.*** 

\section{Implementation details} \label{ID}
***Describe functions and tricks.***

\section{Examples} \label{EG}

We chose our examples from UCSC Genome Bioinformatics Site at 
\url{http://genome.ucsc.edu/} and WormBase at 
\url{http://www.wormbase.org/}.   UCSC Genome Bioinformatics Site  
contains the reference sequence for 
the human and \textit{C. elegans} genomes and working drafts for several organisms 
including chimpanzee, mouse, Drosophila, \textit{C. briggsae}, and  yeast.
WormBase is designed  exclusively for nematodes and provides information 
concerning the genetics,  genomics and biology of \textit{C. elegans} and 
some related nematodes. Currently, WormBase has the reference sequence for  
\textit{C. elegans} and draft sequence for \textit{C.Briggsea}.  As 
more and more nematodes are sequenced, they will be included in the WormBase.

Our first example in Section \ref{wormbase} concerns comparative genomics. Specifically, we 
focus on DNA sequence access for \textit{C. elegans} and \textit{C. briggsea} through WormBase.
The second example in Section \ref{humangenome} focuses on UCSC Genome Bioinformatics Site and 
is especially useful for downstream analysis of human ChIP-Chip data \citep{Cawley04, Keles04.2}.



\subsection{Example I: WormBase} \label{wormbase}

\subsubsection{Extracting  \textit{C. elegans} 
DNA sequence using WormBase.} \label{wormbase.eg1}

This is a rather simple task. The url 
\url{http://www.wormbase.org/db/searches/advanced/dumper} is a
HTML form with fields that one can input specific gene names or
chromosome numbers and specify the type of the sequence of interest,
e.g., feature or flanking sequence. For example, in order to extract 100bp
upstream of the transcription start site for gene unc-47, one has to
input the following to the HTML form.

\begin{itemize}
\item Input ``unc-47'' to the \texttt{Input Options} box.
\item Select the option ``Gene Models'' from the list \texttt{Select one
  feature to retrieve}.
\item Select \texttt{output option} ``flanking sequences only''.
\item Input \texttt{flanking sequence length}, e.g., 100bp 5' flank.
\item Choose \texttt{coordinates relative to}, e.g., chromosome.
\item Choose \texttt{sequence orientation}, e.g., relative to feature.
\item Choose \texttt{output format}, e.g., plain text.
\item Hit the  \texttt{DUMP} button.
\end{itemize}


***Explicit R comments illustrating how we do this by our functions.****

\subsubsection{Extracting DNA sequences for \textit{C. Briggsea} orthologs of
  \textit{C. elegans} genes.} \label{wormbase.eg2}
 This is a more complicated task than the 
  above example  since \textit{C. briggsea} genome is recently sequenced 
\citep{CBgenome} and the information of which  \textit{C. briggsea} genes are
  orthologus to which \textit{C. elegans} genes is not available as 
  a separate look up table in WormBase. One way to extract DNA sequence of the 
  \textit{C. briggsea} ortholog of a \textit{C. elegans} gene is as
  outlined below. This involves firstly extracting the name (sequence id) of the \textit{C. briggsea} ortholog for the gene of interest and then repeating 
steps similar to that of section \label{wormbase.eg1}. 
For illustration purposes, we will use unc-47.

\begin{itemize}
\item Go to  \texttt{Batch Genes} link in the menu bar of WormBase. This
  corresponds to the url
  \url{http://www.wormbase.org/db/searches/info\_dump}.

\item Input  ``unc-47'' to the \texttt{Genes or Loci} box. 

 \item Submitting the above query opens up a HTML page with a hyperlink to
 \texttt{locus} unc-47. Following this link, one obtains a very detailed page
 for unc-47. In particular, this page contains identification,
 location and function information for the gene of interest, e.g.,
 unc-47 in this case. Under the identification title, one has a named hyperlinked  marked as 
 \texttt{Putative C. briggsae ortholog} for the specified gene unc-47.  The name of the hyperlink,  CBG09800,  is a sequence id
 for \texttt{C. briggsea} ortholog of unc-47. This link
 leads to a detailed site for CBG09800 where
 access to the desired \texttt{C. briggsea} sequence id, cb25.fpc2234, that can be used
 for extracting sequences from \texttt{C. briggsea} genome, is possible.

\item As mentioned previously, the above steps are just for the purpose of obtaining the
  proper sequence id of \texttt{C. briggsea} ortholog of a \texttt{C.elegans}
  gene.  Next, one has to go to the \texttt{Batch Sequences} link of
  the WormBase and repeat the steps of Section \ref{wormbase.eg1}  with
  the exception that \textit{C. briggsea} should be selected as the species 
and 
  option ``Integrated (hybrid) briggsea gene set''
  should be chosen  in the \texttt{Select one feature to retrieve}
  field. The output obtained from \texttt{Batch sequences} for
  \texttt{C. briggsea} is also slightly different than the one obtained for \textit{C. elegans}. One typically
  obtains multiple queries for a given sequence id. The partial output obtained for 
  the \textit{C. briggsea} sequence id cb25.fpc2234 is as follows:
\begin{verbatim}
...
>CBG09797 (cb25.fpc2234:223440..227457)
TCTTTTCGCCTCTCTTTTCTTTTTTCCCCCACTATCGTTTTTTTCAGTTACTCCAGTTATCCAT
ATCCTATCTGTAATACTGGACCAGCACTACGGTACA
>CBG09798 (cb25.fpc2234:229315..254784)
CTTTTGCGGACTTGCTGCCAGCCTTCACACAGACCGGCATAACTATATACTACTTCGAGAGATA
GATGAGATTAATTTTTTTCACAACACCCGCATAACA
>CBG09799 (cb25.fpc2234:246992..246175)
AGAATTCATGCAAATCGAATTATGCACAGAAAAGGGAAAAGTATAAAAGAGCGAGTACGGAAGG
CTGAAAATCAGTTTCATTCTTGATTCTCCTCTCGAC
>CBG09800 (cb25.fpc2234:257081..255468)
ACGACGATGACGAGCGCCCAAGAGGTCTCCAGAGCTCTTTTCACAAATTCTCTTCTTTCAAAAC
CGGTGGTTCCTTTCAAGTTTGTGTTTCCTTACAGAC
>CBG09801 (cb25.fpc2234:259453..263924)
CTTTTCTTTTTGGTTTATTCTTCTTCTTTTTTATGTTTGCCACTCTATTTTTAAACTTGTTGCT
TCTATTTTAAACTTTACTACATATATTTCATTTCAG
...
\end{verbatim}
Among these \texttt{>CBG09800 (cb25.fpc2234:257081..255468)} is the one that is orthologus to the unc-47 gene 
of \textit{C.elegans}. 
Note that \texttt{CBG09800} matches the first sequence id that led us to \texttt{cb25.fpc2234}.
*** As mentioned, extracting ortholog sequences from WormBase is pretty tricky. I am sure this will become easier as the 
quality of CB genome increases. For the time being, this is the way to go. As we see, even after extracting the data we have 
to do postprocessing to extract the actual one. Of course, once everything is in the R environment, 
this becomes a trivial task.***



\end{itemize}

***Explicit R comments illustrating how we do this by our functions.****

\subsection{Example II: UCSC Genome Browser} \label{humangenome}

UCSC genome browser provides two main platforms for viewing sequence and related information on several organisms.
The first one is the Human Genome Browser Gateway which provides graphical view of several features of the genomes.
The second one is the Table Browser that mainly provides text output of various features.
Here, we focus on tasks related to Table Browser since this provides output that can be programmatically manipulated in R.
Specifically, as examples, we provide (1) extracting the locations of CpG islands and  known (annotated) genes and (2) extracting  
sequences from specific coordinates on one or several chromosomes.


\subsubsection{Extracting the locations of CpG islands and  known genes  from human genome}
One proceeds as follows in the main page of UCSC Genome Bioinformatics Site:
\begin{itemize}
\item Select \texttt{Tables}. This takes one to  url \url{http://www.genome.ucsc.edu/cgi-bin/hgText}. 
Here, select ``Human'' for \texttt{genome} and ``Nov. 2002'' for \texttt{assembly}, then hit the \texttt{submit} button.
\item In the page that opens next,
\begin{itemize}
\item Choose ``CpGislands''  in the \texttt{Positional Tables} pull down menu. Enter ``chr21'' in the \texttt{position} box. 
Submitting this query by hitting the \texttt{Get all fields} button  creates a table of CpG island locations on chromosome 21.
\item Choose ``KnownGenes'' in the \texttt{Positional Tables} pull down menu. Enter ``chr21'' in the \texttt{position} box.  
Submitting this query by hitting the \texttt{Get all fields} button  creates a list of known genes, including their intron, exon, and location  information,  on chromosome 21.
\end{itemize}
\end{itemize}

***Explicit R comments illustrating how we do this by our functions.****

\subsubsection{Extracting sequences of specific coordinates from human genome}
Text files containing sequences of human chromosomes are available for downloading at the \texttt{download} page of UCSC Genome Bioinformatics Site. 
However, access to sequences of  specific coordinates is also possible through the HTML form ``Add you own custom track'' at url\\
\url{http://genome.cse.ucsc.edu/cgi-bin/hgTracks?org=Human&db=hg16&position=&pix=620&hgsid=32119563&customTrackPage=Add+Your+Own+Custom+Tracks}.\\

\noindent Specifically, one proceeds as follows:

\begin{itemize}
\item Input the desired coordinates into the ``Add you own custom track'' window at the above url. This url also appears as a 
link on the main page of Human Genome Browser Gateway at \url{http://genome.ucsc.edu/cgi-bin/hgGateway}. 
The input should be in a specific 
format as described in the Genome Browser. An example is as follows:


\begin{center}
\begin{tabular}{|l|}
\hline
browser position chr22:40025000-40027001\\
track name=example description="blue ticks are mine" color=0,0,255\\
chr22   40025000 40026000\\
chr22   40026001 40027001\\
\hline
\end{tabular}
\end{center}


These coordinates correspond to bases 40025000 to 40027001 on 22nd human chromosome.

\item Submitting these coordinates, i.e., above text file, 
 automatically takes one to a graphical display of this region at the url 
\url{http://genome.cse.ucsc.edu/cgi-bin/hgTracks}. From here, we follow the  \texttt{Tables} link on the menu bar and this leads 
to  the url

\noindent
\url{http://genome.ucsc.edu/cgi-bin/hgText?db=hg16&position=chr22:40025000-40027001&phase=table&tbPosOrKeys=pos&hgsid=34125104}.

\noindent
Note that this url is coordinate-specific whereas the graphical display url is not.
Now, \texttt{custom tracks} pull down menu on this page has an option ``ct\_example''. After selecting this, we click on 
\texttt{Get sequence}. This leads to  a form at the coordinate-specific  url  

\noindent
\url{http://genome.ucsc.edu/cgi-bin/hgText?hgsid=34125104&db=hg16&tbTrack=Choose+table&tbCustomTrack=customTrack.ct_example&table0=Choose+table&table1=Choose+table&tbPosOrKeys=pos&position=chr22\%3A40\%2C025\%2C000-40\%2C027\%2C001&origPhase=table&phase=Get+sequence...}

\noindent where \texttt{Sequence Retrieval Region Options} and 
\texttt{Sequence Formatting Options} are presented. This forms allows one to require additional base pairs from upstream or 
downstream of the original coordinates. Finally, hitting the \texttt{Get sequence} button, one obtains 
the sequences of the  regions specified in the custom track ``ct\_example''. 
\scriptsize
\begin{verbatim}
>hg16_ct_example_(null) range=chr22:40025001-40026000 5'pad=0 3'pad=0 revComp=FALSE strand=
TACAGGTGTGAGCCACCGCCCCCGGTACCTGGCTAATTTTTGTATTTTTA
GGAGAGACAGGGTTTCATCATGTTGGCCTGGCTGGTCTCAAACTCCTGAC
CTCGGGATCCACTCGCCTGGGCCTCTCAAAGTACTGGGATTGCAATCCTG
ACCCCCCCGCACCTGGCCAGTGCTACAGGCTATTCTGAATTAACCCTGTG
GCCAGGTGCAGTGGCTCACGCCTGTAATCCCATCACTTTGGGAGGCCAAG
GCAGGTGGATCACCTGAGGTCAGAAGTTCGAAACCAGTCTGGCCAACATG
TTGAAACCCCAGTCTCTACTAAATACAAAAAAAATTAATCAGGCGTGGTG
CCGCATGCCTATAATCCCAGCTACTTGGGAGGCTGAGGCAGGAGAATCGC
TTGAACCAGGGAGGCGGAGGTTGCAGTGAGCCTAGATTGTGCCATTGCAC
TCCAGCCTGGGCAACAGAGCGAAGCTCCATCTCAAAAAAAAAAATAAAAC
AGGCTGGGCGTGGTGGCTCATGCCTGTAATCCCAGTACTTTGGGAGGCTG
AGGCGGGTGGATCACCTGAGGTCAGGAGTTTGAGACCAGTCTGGCCAACA
TGGTGAAACTCCGTCTCTACTAAAAATACAAAAAATTAGCTGGTCATGGT
GGCAGGCTCCTGTAATCCCAGTTTACTGGGGAGACTGAGACAGGAGAATT
GCTTGAACCCAGGAGGCAGAGGTTGCAGTGAGCCAAGATTGCGCCATTGT
AAGCCAGCCGAAGCAACAAAAGTGAAACTCTGTCTCAAATAAATAAATAA
AATAAAATAAAACAGGCCAGGCATAGTGGCTCATGCCTGTAATCCCAGCA
CTTTGGTGGGTGGATCACCTGAGCTCAGGAGTTCGAGACCAGCCTGGCCA
ACATAGTAAAACCCCATCTCTACTAAAAATACAAAAAATAGCTGGGCATG
GTGGTGCGTACCTGTAATCCCAGCTACTCGGGAGTCTGAGGCACAAGAAT
>hg16_ct_example_(null) range=chr22:40026002-40027001 5'pad=0 3'pad=0 revComp=FALSE strand=
GCTTGAACCCAGGAGGTGGAGGTTGCAGCGAGCCGAAATTGGGTCACTGC
ACTCCAGCCTGGGCGACGAGCAAAATACTGTCTCAAAATAATAAAAACTA
AAATAAAATTAAATAATGAATTAACCCTATGACCGCTTAAGGTCTCAGAG
TATGTGGTAAGCCTTCTGGGAAGGCCAGGCATGGATGAGGATGGGATGCC
GTGTCTAAAGGAAAACAGCCGGCCTGGTTCAAGTGCGAGATTCGAACTGG
AGAATAGAGTAGGTTCTTGAAATGCATGGGTTGGAATCGGGGCTCTGAGA
AAGATGGGCCAGGCCTGTGAGGCTCACTGCCCACTTCCTGGGATTGAGTT
ACTCTCCTGTGTGGTATTTCATCGCAGATTTTGTTTCTGCAGATAAGGAA
AAGGGGAAGGAAAAGCTGGAGGAGGACGAGGCCGCAGCCGCCAGCACCAT
GGCTGTCTCAGCCTCCCTCATGCCACCCATCTGGGACAAGACCATCCCAT
ATGATGGCGAATCTTTCCACCTGGAGTACATGGACCTGGATGAGTTCCTG
CTGGAGAATGGCATCCCCGCCAGCCCCACCCACCTGGCCCACAACCTGCT
GCTGCCTGTAGCAGAGCTAGAAGGGAAGGAGTCTGCCAGCTCTTCCACAG
CATCCCCACCATCCTCCTCCACTGCCATCTTTCAGCCCTCTGAAACCGTG
TCCAGCACAGGTTGGTGAAAGGCCATCGAGGAGGGCCACCTGTCCCATCC
AGGGAAGTCCATTTCCCACTGAGGGCTGCTTGTGTCCATGTACCCAGTGA
GTCCCTTCTCTGAGGGGAGGTCCTGGTGGGGCTGCAGTGTGGAGAAACTC
TTCACTCCCCACCCTTCCACACAGTTCCCCGAGGCTGGAGATAAAAATAG
TGGTCATGTCACTGGCAGTTGGAACCCTTCTGTTGGTTGCTGTGCTGCAA
ACAGTCAGGACAGGACCTTCCAGGCCACGGCATTACAGGATAGAAATCCC
\end{verbatim}
\normalsize


\end{itemize}

***Explicit R comments illustrating how we do this by our functions.****

\section{Summary and discussion}

\bibliographystyle{dcu}

\bibliography{accessHTML.bib}
\end{document}
