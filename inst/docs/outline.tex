\documentclass{article}
\usepackage{fullpage}
\usepackage{times}
\def\HTMLTag#1{<#1>}
\def\HTMLAttr#1{\textbf{#1}}
\def\SPackage#1{\textit{#1}}
\usepackage{hyperref}

\title{Automating Access to Data from HTML Forms}
\author{Duncan Temple Lang}
\begin{document}
\maketitle

Downloading data via HTML forms is tedious and error prone when done
repeatedly.  It is better to access the data programmatically in order
to allow automated access that can be easily controlled in high-level
languages.  Additionally, such access allows particular data to be
determined dynamically within a script and then accessed.

To provide such programmatic access, we need to emulate the HTML form
access.  With that interactive interface, the user completes different
elements of the HTML form and submits the query to the Web server.  On
submission in the user's browser, the contents of the different form
elements are collected and sent as a query to the target Web server.
The query combines the name and value for each form element, including
both visible and \textit{hidden} fields.  There are details about how
the query is constructed, ecoded and sent to the Web server.

A natural equivalent interface in a progarmming language like R is to
have an R function that allows the caller to specify the same
information as she might on the HTML form.  At its simplest (and most
common), each visible HTML form element corresponds to a formal
argument of the R function.  For HTML elements with default values, we
can use those same default values in R.  Hidden HTML form elements are
not included in the formal arguments but are sent as part of the
query.  The Submit button of the HTML form corresponds to the actual
invocation of the function and so is not an explicit part of the
function.

If we wish to create an R function corresponding to 
an HTML form, we can examine the rendered HTML form and
the corresonding HTML source.  
We find the different form elements such as
\HTMLTag{input},
\HTMLTag{textarea}, \HTMLTag{select}, \HTMLTag{option}.
For each element, we take the name 
and define a formal argument for the R function.
We also determine if the element has a default value.
And for elements such as option menus,
we compute the set of permissible values.

Let's look at a simple example
from  \href{http://www.genome.ucsc.edu/cgi-bin/hgText}{the UC Santa
  Cruz Genome web site}.
This has been explicitly formatted
to make it easier to read and identify the different
elements.
\begin{figure}%[htbp]
  \begin{center}
    \leavevmode
\begin{verbatim}
 <form action="/cgi-bin/hgText" name="mainForm" method="post">
  <table bgcolor="#cccc99" border="0" cellpadding="1" cellspacing="0">
   <tbody>
    <tr>
     <td>
      <table bgcolor="#fefdef" bordercolor="#cccc99" border="0" cellpadding="0" cellspacing="0">
       <tbody>
        <tr>
         <td>
          <table bgcolor="#fffef3" border="0">
           <tbody>
            <tr>
             <td>
              <input type="hidden" name="hgsid" value="30932773">
              </input>
              <input type="image" border="0" name="tbDummyEnterButton" src="hgText_files/DOT.gif">
              </input>
              <table>
               <tbody>
                <tr>
                 <td align="center" valign="baseline">
                 genome
                 </td>
                 <td align="center" valign="baseline">
                 assembly
                 </td>
                </tr>
                <tr>
                 <td align="center">
                  <select name="org" onchange="document.orgForm.org.value = document.mainForm.org.options[document.mainForm.org.selectedIndex].value; document.orgForm.db.value = 0; document.orgForm.submit();">
                   <option selected="selected" value="Human">
                   Human
                   </option>
                   <option value="Chimp">
                   Chimp
                   </option>
                   <option value="Mouse">
                   Mouse
                   </option>
                   <option value="Rat">
                   Rat
                   </option>
                   <option value="Chicken">
                   Chicken
                   </option>
                   <option value="Fugu">
                   Fugu
                   </option>
                   <option value="Fruitfly">
                   Fruitfly
                   </option>
                   <option value="C. elegans">
                   C. elegans
                   </option>
                   <option value="C. briggsae">
                   C. briggsae
                   </option>
                   <option value="Yeast">
                   Yeast
                   </option>
                   <option value="SARS">
                   SARS
                   </option>
                   <option value="Zoo">
                   Zoo
                   </option>
                  </select>
                 </td>
                 <td align="center">
                  <select name="db">
                   <option selected="selected" value="hg16">
                   July 2003
                   </option>
                   <option value="hg15">
                   April 2003
                   </option>
                   <option value="hg13">
                   Nov. 2002
                   </option>
                  </select>
                 </td>
                 <td>
                  <input type="submit" name="submit" value="Submit">
                  </input>
                 </td>
                </tr>
               </tbody>
              </table>
             </td>
            </tr>
           </tbody>
          </table>
         </td>
        </tr>
       </tbody>
      </table>
     </td>
    </tr>
   </tbody>
  </table>
  <input type="hidden" name="phase" value="table">
  </input>
 </form>
\end{verbatim}
    \caption{UCSC Genome Form}
    \label{fig:ucscform}
  \end{center}
\end{figure}

Unfortunately, much of this is formatting information rather than
related to the actions of the form.
The important components of the form are 
\begin{enumerate}
\item \verb+ <form action="/cgi-bin/hgText" name="mainForm"  method="post">+
\item \verb+ <input type="hidden" name="hgsid" value="30932773">+
\item \verb+ <input type="image" border="0" name="tbDummyEnterButton" src="hgText_files/DOT.gif">+
\item \verb+<select name="org">+ and the associated \HTMLTag{option}
  elements such as \verb+<option selected="selected" value="Human">+
\item \verb+ <input type="hidden" name="phase" value="table">+
\end{enumerate}

Essentially, there are two entries of interest: the two pull down
menus. These are identifed by the \HTMLTag{select} tags and are
populated with choices via the nested \HTMLTag{option} tags.
As with all form elements, each 
\HTMLTag{select} entry has a name.
In our example, these are  \texttt{org} and \texttt{db}.
These become the names of the formal arguments in our
R function.
Since, for each of these menus, the user of the HTML form can only select
a value from within its list, we know the set of possible values.
For example, looking at the \texttt{db} element\footnote{The value
\texttt{db} comes from the \HTMLAttr{name} attribute of the
\HTMLTag{select} element.}, we
have three possible values:
 July 2003, April 2003, and  Nov. 2002.
Corresponding to these visible selections
are the three values that will be sent as
the actual selected value in the HTTP query
when the form is submitted. These are 
"hg16", "hg15" and "hg13".
From the HTML source, we see that the 
default value is July 2003 (i.e. hg16) as that
has the additional HTML attribute 
\HTMLAttr{selected}.
To provide an interface to this argument in R,
we might define our function something like
\begin{verbatim}
hgText =
function(db = "hg16") 
{
 if(!missing(db)) {
   dbValues =  c("hg16" = "July 2003",
                 "hg15" = "April 2003",
                 "hg13" = "Nov. 2002")

   
   i = pmatch(db, names(dbValues))
   if(is.na(i)) {
      i = pmatch(db, dbValues)
      if(is.na(i))
         stop("Incorrect value for db:", db, " must be one of ",  
                 paste(names(dbValues), collapse= ", "))
      
      db = db[i]
   } else 
      db = names(dbValues)[i]
 }
}
\end{verbatim}

Obvsiously, we need to extend this to include the
\texttt{org} \HTMLTag{select} element in this function.
And we need to include the hidden elements
(\texttt{phase} and \texttt{hgsid}) in the form in
the query and actually send the query.

We can construct such functions manually by examining the HTML source
for the form.  However, we can also do this programmatically by
parsing the HTML content and extracting the relevant information for
the different form element types.  We can do this in R using the HTML
parsing mechanism in the \SPackage{XML} package.  By providing code to
process each of the different HTML form element types, we can extract
the name of the form element, its default value, its possible values
(i.e. options) and readily automate the task of constructing the R
function to mimic the HTML form.





\end{document}
